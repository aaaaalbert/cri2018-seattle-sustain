%!TEX root = preliminary_proposal.tex

\section{Project Description}

\explain{Two pages overall}

\textbf{CRI Pre-proposal: CI-SUSTAIN: CNS: \proposalname}

\textbf{Lead institution:} Department of Computer Science and Engineering, NYU, Justin Cappos (PI) and Albert Rafetseder (Co-PI)

\textbf{Institutions submitting collaborative proposals:} None

\textbf{Other collaborators:} No External Collaborators

\textbf{Projected budget total (total of all collaborative pieces):} \budget

\textbf{Keywords:} peer-to-peer testbed, community participation,
collaborative experimentation


\subsection{Overview}
\explain{A concise description of the infrastructure to be developed, enhanced, or
sustained. This includes a description of major equipment needs for the 
project as well as other significant costs. Projects that involve 
enhancements to, or sustainment of, existing infrastructures should 
include information about the existing resources.}

\sysname is an open, participatory network testbed used in research
and education. Despite receiving only limited and intermittent
funding since its inception in 2008, \sysname has scaled to attract
more than 40,000 installs on machines all around the world, has
provided resources to more than 4,000 researchers in the academic
community, was used in about 100 classes (and is still used), and
resulted in dozens of papers using the platform.
Furthermore, a vibrant community has sprouted around the system:
Volunteers from over 30 institutions have contributed code,
documentation, or deployed instances of \sysname services.
There is a considerable user base both in the United States
and abroad, including distant places such as Sweden, Austria, and India.

Despite its successes, some operational challenges associated with
keeping the system running and up-to-date are bigger than what the
current \sysname community can contribute on a purely volunteer
basis.
For instance, \sysname is based on a renowned open-source software
stack that includes the scripting language Python, the Django Web
framework, and the Apache web server software. New versions of
these dependencies bring new functionalities and require careful
modifications to the existing codebase to ensure that existing
workflows keep functioning, and new features are added to the
system to benefit the community of users.

The same can be said for the computers that the \sysname core
team operates at NYU. The largest proportion of testing on new
operating systems and hardware is done and contributed to the
community as a token of good will, without funding from the project.

\todo{Describe the solution(s) to the problem.}


\subsection{CISE Research Focus}
\explain{This section describes the research focus that is enabled by the 
infrastructure, the importance of the research problems to advancing 
CISE research frontiers, and the expertise of the research team relative 
to the focused research thrust. The description should identify the 
project team and detail each member’s contributions to the project as 
well as specific expertise relative to the proposed focused research 
agenda.}

%\sysname provides to the community a collaborative and participatory
%platform for large-scale distributed sensing experiments. This
%advances the understanding and fosters the creation of new
%approaches to designing, operating and maintaining distributed
%collaborative infrastructures. Thanks to its openness, \sysname
%allows for research across a wide range of topics and disciplines,
%from the construction and integration of sensor-enabled Things
%to community-run sensing campaigns and collective data analysis.
%Enabling these approaches is ever more so important because
%existing systems do not provide comparable levels of openness,
%inhibit collaboration, and cannot reach the scale and distribution
%envisioned for \sysname.

PI Justin Cappos has a long history of involvement in
community-supporting CISE infrastructure: He created the \textsc{Stork}
package manager included in the renowned Planet-Lab network testbed,
and the peer-to-peer Seattle testbed platform.
Many of PI Cappos' current projects address practical security issues
in software supply chains, including those for firmware in cars and
medical devices. For \sysname, PI Cappos will oversee the development
and assist in the dissemination of the system and its use.

PI Albert Rafetseder is the current technical lead for the Seattle
testbed. He ported the system to lower-power devices such as WiFi
routers and smartphones, thus helping to tap these devices as
resources for research.
His academic contributions similarly target large-scale, distributed
network measurements and sensing campaigns. PI Rafetseder is an avid
contributor to and advocate of open-source software and Open Data.
For \sysname, PI Rafetseder will lead the design and implementation
processes, and accommodate community contributions to the system.



\subsection{Sample research project}
\explain{A short (2-3 sentences) summary of one potential research project 
should be included.}

A research group investigates noise levels and air quality in
urban and countryside settings. Existing \sysname installations
provide a simple-to-use system for initial examinations, creating
a reasonable set of base data to motivate more detailed studies.
Later on, the group develops and deploys their own calibrated
high-accuracy sensors, and contributes them to the \sysname network
to spur further participation.



\subsection{Nature of the community involvement}
\explain{This 
section demonstrates the community involvement in the creation, 
enhancement, evaluation, and use of the resource. Describe the research 
community involved in the project. CI-New projects should show the 
community involvement and demand for the project as well as indicate 
the community commitment to establishing and using the infrastructure. 
}



\subsection{Relevance to CISE}
\explain{
This section should include:
\begin{itemize}
  \item A list of specific CISE researchers who are involved in the leadership of the project and in the development of the infrastructure (including any enhancement or sustainment plan, as appropriate);
  \item A list of the CISE researchers communities that will benefit from the infrastructure; and
  \item A list of any prior CISE funding the infrastructure has received (enhancement projects should also include the approximate date when the infrastructure involved was established).
\end{itemize}
}

The project is led by PIs Cappos and Rafetseder.

\sysname will provide benefits to communities in all CISE disciplines:
It is an ideal platform for developing new Computer Science
foundations for sensor technology and distributed systems;
Information Sciences benefit from the novel ways of gathering
and evaluating sensor data with unprecedented methodological
diversity; % mention privacy too?
\sysname's multidisciplinary attitude aligns well with the
Engineering Sciences' and provides a useful tool for the field.

\sysname has received prior CISE funding in grants \todo{ADD PREVIOUS SEATTLE GRANTS}.